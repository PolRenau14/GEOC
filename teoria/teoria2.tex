\documentclass{article}
\usepackage[utf8]{inputenc}
\usepackage{amsmath}
\usepackage{graphicx}
\graphicspath{ {./Grafics/}}

\title { Teoria Geoc: Sweep-line algorithm}
\date { Quadrimestre Tardor 2019-2020}
\author {Pol Renau Larrod\'e}

\begin{document}
  \maketitle
  \newpage

  \section {Intersecting line-segment}

  Important observation of sweep-line.


  Observation1
  \\
  If two line-segments have disjoint projections onto a given line, then they're disjoint.

  \\

  Observation2
  \\
  whern sweepomg a set of line-segments with a line, two Intersecting line-segments need to be consecutive in the sweeping line right before their intersection point.

  \section{Sweep-line algorithm}

  A straight line scans the scene, and allows detecting and constructing the desired elements, leaving the problems solved behind it. the sweeping proces is discretized:

  \\
  Essential elements:

  * Sweep line:
    Data structure storing the information of the portion of the scene intersected by the sweep line. Still updated all time.

  * Events queue:
    Priority queue keeping the information of the locations where the sweepline changes and needs to be updated. At most of times (start point and endpoints)

  * Output data structure

\end{document}
